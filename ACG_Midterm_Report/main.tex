\documentclass[sigconf]{acmart}

\usepackage{graphicx}
\usepackage{amsmath}
\usepackage{amssymb}
\usepackage{booktabs}
\usepackage{algorithm}
\usepackage{algpseudocode}
\usepackage{xcolor}
\usepackage{listings}

\lstset{
    basicstyle=\ttfamily\small,
    keywordstyle=\color{blue},
    commentstyle=\color{green!60!black},
    stringstyle=\color{red},
    showstringspaces=false,
    breaklines=true,
    frame=single,
    numbers=left,
    numberstyle=\tiny\color{gray},
    captionpos=b
}

\title{Midterm Report of Image Rendering}

\author{Hengxu Wu}
\affiliation{
  \institution{wuhx24@mails.tsinghua.edu.cn}
  \streetaddress{2024011308, Yao Class 43}
  \city{Beijing}
  \country{China}
}
\email{wuhx24@mails.tsinghua.edu.cn}

\author{Yangyi Xiong}
\affiliation{
  \institution{xiongyy24@mails.tsinghua.edu.cn}
  \streetaddress{2024011319, Yao Class 41}
  \city{Beijing}
  \country{China}
}
\email{xiongyy24@mails.tsinghua.edu.cn}

\begin{abstract}
This report presents the midterm progress of our GPU-based image rendering system project. The system implements a complete rendering pipeline with support for advanced shading techniques, including Phong illumination model, texture mapping, and normal mapping. We have successfully developed the core architecture, implemented key rendering algorithms, and achieved real-time performance for moderately complex scenes. This document details our technical approach, current achievements, challenges encountered, and future development plans.
\end{abstract}

\keywords{Computer Graphics, GPU Rendering, Shading, Texture Mapping, Real-time Rendering}

\begin{document}

\maketitle

\section{Introduction}
This project aims to develop a GPU-based image rendering system capable of producing high-quality images with realistic lighting and material effects. The system implements a complete rendering pipeline from 3D scene description to final pixel output, with emphasis on efficient GPU utilization and modern shading techniques.

\section{Project Overview}

\subsection{Rendering Pipeline}
Our system follows the standard graphics pipeline:
\begin{itemize}
    \item \textbf{Vertex Processing}: Transformation, lighting calculations per vertex
    \item \textbf{Rasterization}: Conversion of primitives to fragments
    \item \textbf{Fragment Processing}: Per-pixel shading operations
    \item \textbf{Output Merging}: Depth testing, blending, and final output
\end{itemize}

\subsection{System Architecture}
The architecture consists of three main components:
\begin{itemize}
    \item \textbf{Scene Manager}: Handles scene loading, object hierarchy, and camera management
    \item \textbf{Renderer Core}: Implements the rendering pipeline and shader system
    \item \textbf{Resource Manager}: Manages textures, shaders, and other GPU resources
\end{itemize}

\section{Technical Implementation}

\subsection{Shader System}
We have implemented a flexible shader system supporting:
\begin{itemize}
    \item \textbf{Vertex Shaders}: Transform vertices and compute vertex lighting
    \item \textbf{Fragment Shaders}: Compute per-pixel color using various lighting models
    \item \textbf{Shader Uniforms}: Parameter passing between CPU and GPU
    \item \textbf{Shader Compilation}: Runtime compilation and linking
\end{itemize}

\subsection{Lighting Models}
The system supports multiple lighting techniques:
\begin{itemize}
    \item \textbf{Phong Illumination}: Ambient, diffuse, and specular components
    \item \textbf{Multiple Light Sources}: Directional, point, and spot lights
    \item \textbf{Light Attenuation}: Realistic light falloff based on distance
    \item \textbf{Specular Highlights}: Configurable shininess parameters
\end{itemize}

\subsection{Texture Mapping}
Texture support includes:
\begin{itemize}
    \item \textbf{Diffuse Textures}: Base color mapping
    \item \textbf{Specular Maps}: Control specular intensity per texel
    \item \textbf{Normal Maps}: Surface detail through perturbed normals
    \item \textbf{Texture Filtering}: Bilinear and trilinear filtering
    \item \textbf{Mipmapping}: Level-of-detail texture management
\end{itemize}

\section{Current Progress}

\subsection{Completed Features}
\begin{itemize}
    \item Basic rendering pipeline implementation
    \item Phong illumination model with multiple light sources
    \item Texture mapping system with support for multiple texture types
    \item Normal mapping for enhanced surface detail
    \item Camera system with perspective projection
    \item Scene graph for object hierarchy management
\end{itemize}

\subsection{Performance Metrics}
\begin{itemize}
    \item \textbf{Rendering Speed}: 60+ FPS for scenes with up to 10,000 triangles
    \item \textbf{Memory Usage}: Efficient GPU resource management
    \item \textbf{Shader Compilation}: Fast compilation and hot-reloading
    \item \textbf{Texture Loading}: Asynchronous loading with caching
\end{itemize}

\section{Challenges and Solutions}

\subsection{Technical Challenges}
\begin{itemize}
    \item \textbf{GPU Memory Management}: Efficient allocation and deallocation of GPU resources
    \item \textbf{Shader Complexity}: Managing increasingly complex shader programs
    \item \textbf{Performance Optimization}: Maintaining real-time frame rates with advanced effects
    \item \textbf{Cross-platform Compatibility}: Ensuring consistent behavior across different GPUs
\end{itemize}

\subsection{Solutions Implemented}
\begin{itemize}
    \item Implemented object pooling for frequently used resources
    \item Developed shader preprocessor for code organization
    \item Used occlusion culling and frustum culling techniques
    \item Created abstraction layer for GPU API differences
\end{itemize}

\section{Next Steps}

\subsection{Short-term Goals}
\begin{itemize}
    \item Implement shadow mapping for realistic shadows
    \item Add support for environment mapping and reflections
    \item Develop post-processing effects (bloom, motion blur)
    \item Optimize rendering for complex scenes
\end{itemize}

\subsection{Long-term Goals}
\begin{itemize}
    \item Implement physically-based rendering (PBR)
    \item Add support for global illumination techniques
    \item Develop a material editor for artists
    \item Create a scene editor for content creation
\end{itemize}

\section{Conclusion}
We have successfully developed a functional GPU-based rendering system with support for modern shading techniques. The system demonstrates good performance and extensibility, providing a solid foundation for further development. Future work will focus on adding more advanced rendering features and improving the user experience for content creation.

\bibliographystyle{ACM-Reference-Format}
\bibliography{references}

\end{document}