\documentclass[sigconf]{acmart}

\usepackage{graphicx}
\usepackage{amsmath}
% 避免 \Bbbk 重复定义冲突
\let\Bbbk\relax
\usepackage{amssymb}
\usepackage{booktabs}
\usepackage{algorithm}
\usepackage{algpseudocode}
\usepackage{xcolor}
\usepackage{listings}

\lstset{
    basicstyle=\ttfamily\small,
    keywordstyle=\color{blue},
    commentstyle=\color{green!60!black},
    stringstyle=\color{red},
    showstringspaces=false,
    breaklines=true,
    frame=single,
    numbers=left,
    numberstyle=\tiny\color{gray},
    captionpos=b
}

\title{Midterm Report of Image Rendering}

\author{Hengxu Wu}
\affiliation{
  \institution{2024011308, Yao Class 43}
  \city{Beijing}
  \country{China}
}
\email{wuhx24@mails.tsinghua.edu.cn}

\author{Yangyi Xiong}
\affiliation{
  \institution{2024011319, Yao Class 41}
  \city{Beijing}
  \country{China}
}
\email{xiongyy24@mails.tsinghua.edu.cn}

\begin{abstract}
This report presents the midterm progress of our GPU-based image rendering system project. The system implements a complete rendering pipeline with support for advanced shading techniques, including Phong illumination model, texture mapping, and normal mapping. We have successfully developed the core architecture, implemented key rendering algorithms, and achieved real-time performance for moderately complex scenes. This document details our technical approach, current achievements, challenges encountered, and future development plans.
\end{abstract}

\keywords{Computer Graphics, GPU Rendering, Shading, Texture Mapping, Real-time Rendering}

\begin{document}

\maketitle

\section{Introduction}
This project aims to architect a high-performance rendering system purely on GPU using DirectX12. Unlike traditional approaches that rely on pre-existing frameworks, we are building the entire pipeline from scratch to ensure maximum control and optimization.

Our system supports multiple model formats including OBJ, glTF, FBX, and Blend. It features a GPU-based path tracing renderer that implements acceleration structures, material evaluation, and light sampling. For image quality improvement, we have integrated Intel Open Image Denoise~\cite{OpenImageDenoise} and are developing special effects such as Depth of Field and Motion Blur.

\section{Implementation}

\subsection{Overall Architecture}
The system architecture is designed to be modular and efficient. As shown in Figure~\ref{fig:arch}, the pipeline is divided into three main stages: Conversion, Rendering, and Processing.

\begin{figure}[h]
    \centering
    \includegraphics[width=0.9\linewidth]{../slide/fig/arch.png}
    \caption{System Architecture Diagram}
    \label{fig:arch}
\end{figure}

We have made significant progress in implementing various features. The material system supports basic diffuse, specular, and transmissive materials, as well as Principled BSDF and multiple-layer materials. The texture system currently supports basic color textures. The scene management system can handle various formats and render a huge amount of resources available online, as demonstrated in the Results section. Additionally, antialiasing is supported through random jittering.

\subsection{Workflow}

\subsubsection{Conversion}
The conversion stage involves loading models in various formats using our Python-based converter. These models are then converted into our custom binary format, the ACG file (\texttt{.acg}). This process ensures that complex materials, such as those from Blender's Principled BSDF shader, are correctly preserved and optimized for our renderer.

\begin{figure}[h]
    \centering
    \includegraphics[width=0.9\linewidth]{../slide/fig/arch_convert.png}
    \caption{Conversion Workflow}
    \label{fig:convert}
\end{figure}

\subsubsection{Rendering}
In the rendering stage, the system loads the ACG files and creates the necessary GPU resources. We have implemented a GPU-based path tracing renderer using DirectX12. This renderer handles complex material evaluation, light sampling algorithms, and skybox lighting.

\begin{figure}[h]
    \centering
    \includegraphics[width=0.9\linewidth]{../slide/fig/arch_render.png}
    \caption{Rendering Workflow}
    \label{fig:render}
\end{figure}

\subsubsection{Processing}
The processing stage focuses on enhancing the final image quality. We have integrated Intel Open Image Denoise to provide high-quality denoising. Other special effects, such as Depth of Field and Motion Blur, are currently in development. The final image output and saving functionalities are also handled in this stage.

\begin{figure}[h]
    \centering
    \includegraphics[width=0.9\linewidth]{../slide/fig/arch_process.png}
    \caption{Processing Workflow}
    \label{fig:process}
\end{figure}

\subsection{GUI}
To facilitate user interaction, we have developed a Graphical User Interface (GUI) as shown in Figure~\ref{fig:gui}. The GUI allows users to adjust render settings such as output resolution, sampling parameters, lighting intensity, and scene model paths. Camera settings, including position, target, up vector, and field of view, can also be configured. The GUI displays the rendered image in real-time and provides controls to start or stop the rendering process. Log details are also available for debugging purposes.

\begin{figure}[h]
    \centering
    \includegraphics[width=0.9\linewidth]{../slide/fig/gui.png}
    \caption{GUI Screenshot}
    \label{fig:gui}
\end{figure}

\section{Results}

\subsection{Test Scene \& Denoising}
We used the Cornell Box as a test scene to evaluate our denoising implementation. Figure~\ref{fig:denoise} compares the results with and without denoising, both using 100 samples per pixel. The denoising process is highly efficient, taking less than 1 second for a 1920x1080 image, making it suitable for practical use.

\begin{figure}[h]
    \centering
    \includegraphics[width=0.9\linewidth]{../slide/fig/denoise.png}
    \caption{Denoising Effect: Left - Denoised, Right - Noisy}
    \label{fig:denoise}
\end{figure}

\subsection{Advanced Materials}
Our renderer successfully supports complex materials, including glass, metal, and layered materials. Figure~\ref{fig:cubes} shows different complex material cubes rendered in the scene, demonstrating the system's capability to handle sophisticated material properties. Additionally, Figure~\ref{fig:cubes2} illustrates the accurate rendering of Blender's Principled BSDF material.

\begin{figure}[h]
    \centering
    \includegraphics[width=0.8\linewidth]{../slide/fig/cubes.png}
    \caption{Advanced Materials: Complex material cubes}
    \label{fig:cubes}
\end{figure}

\begin{figure}[h]
    \centering
    \includegraphics[width=0.9\linewidth]{../slide/fig/cubes2.png}
    \caption{Advanced Materials: Blender's Principled BSDF}
    \label{fig:cubes2}
\end{figure}

\subsection{Large Scene}
We have also tested the renderer with large scenes. Figure~\ref{fig:large_scene} shows the San-Miguel model, which contains 1.7 million triangles and over 200 textures. By using virtual textures, we can handle large scenes with high-resolution textures without overwhelming GPU memory.

\begin{figure}[h]
    \centering
    \includegraphics[width=0.8\linewidth]{../slide/fig/large_scene.png}
    \caption{Large Scene Rendering: San-Miguel Model}
    \label{fig:large_scene}
\end{figure}

\subsection{Skybox \& Other Formats}
Figure~\ref{fig:cirno} demonstrates our renderer's capability to utilize skybox images for realistic environmental lighting. The scene features a 3D-scan Cirno fumo model. We have also successfully rendered models from various formats such as glTF and FBX, showcasing the versatility of our system.

\begin{figure}[h]
    \centering
    \includegraphics[width=0.8\linewidth]{../slide/fig/cirno.jpg}
    \caption{Skybox Lighting: 3D-scan Cirno fumo scene}
    \label{fig:cirno}
\end{figure}

\subsection{Other Show Cases}
We present additional rendering results to demonstrate the versatility of our system. Figure~\ref{fig:demo2} shows a rendered Breakfast Room scene. Figure~\ref{fig:demo3} displays the Cornell Box with a mirror, highlighting reflection capabilities. Figure~\ref{fig:demo4} presents the Sponza scene, a classic test model for global illumination.

\begin{figure}[h]
    \centering
    \includegraphics[width=0.9\linewidth]{../slide/fig/demo2.png}
    \caption{Scene: Breakfast Room}
    \label{fig:demo2}
\end{figure}

\begin{figure}[h]
    \centering
    \includegraphics[width=0.9\linewidth]{../slide/fig/demo3.png}
    \caption{Scene: Cornell Box with Mirror}
    \label{fig:demo3}
\end{figure}

\begin{figure}[h]
    \centering
    \includegraphics[width=0.9\linewidth]{../slide/fig/demo4.png}
    \caption{Scene: Sponza}
    \label{fig:demo4}
\end{figure}

\section{Future Plan}
In the remaining time of the course, we plan to implement the following features:
\begin{itemize}
    \item \textbf{Support for Blender and FBX formats}: Expected to be completed within 1 week.
    \item \textbf{Volumetric Rendering}: Planned for the next 1-2 weeks.
    \item \textbf{Special effects}: Such as Depth of Field and Motion Blur, planned for 2-3 weeks.
    \item \textbf{Final optimizations and testing}: Will be conducted in the remaining time.
\end{itemize}

\section{Conclusion}
We have successfully developed a functional GPU-based rendering system with support for modern shading techniques. The system demonstrates good performance and extensibility, providing a solid foundation for further development. Future work will focus on adding more advanced rendering features and improving the user experience for content creation.

\bibliographystyle{ACM-Reference-Format}
\bibliography{references}

\end{document}