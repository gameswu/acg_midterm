\documentclass[acmtog]{acmart}

\usepackage{booktabs} % For better tables
\usepackage{graphicx}
\usepackage{subcaption}
\usepackage{amsmath}
\usepackage{algorithm}
\usepackage{algpseudocode}
\usepackage{xcolor}
\usepackage{listings}
\usepackage{hyperref}

\title{Midterm Report of Image Rendering}
\author{Hengxu Wu}
\affiliation{%
  \institution{Shanghai Jiao Tong University}
  \city{Shanghai}
  \country{China}}
\author{Yangyi Xiong}
\affiliation{%
  \institution{Shanghai Jiao Tong University}
  \city{Shanghai}
  \country{China}}

\begin{document}

\begin{abstract}
This report presents the midterm progress of our GPU-based image rendering system project. We have implemented a basic rendering pipeline with support for vertex processing, primitive assembly, rasterization, and fragment shading. The system currently supports basic lighting models, texture mapping, and simple scene management. We discuss the technical challenges encountered, our solutions, and the planned improvements for the next phase of development.
\end{abstract}

\maketitle

\section{Introduction}
Image rendering is a fundamental task in computer graphics, transforming 3D scene descriptions into 2D images. Our project aims to develop a GPU-based rendering system that efficiently processes complex scenes with realistic lighting and material properties. This midterm report summarizes our progress, technical implementation details, and future plans.

\section{Project Overview}
Our rendering system is designed as a modular GPU pipeline with the following key components:

\subsection{Rendering Pipeline}
The pipeline follows the standard graphics pipeline architecture:
\begin{itemize}
    \item \textbf{Vertex Processing}: Transform vertices from model space to clip space
    \item \textbf{Primitive Assembly}: Assemble vertices into geometric primitives (triangles)
    \item \textbf{Rasterization}: Convert primitives into fragments (pixels)
    \item \textbf{Fragment Shading}: Compute color for each fragment
    \item \textbf{Output Merging}: Combine fragments into final image
\end{itemize}

\subsection{System Architecture}
The system is implemented using modern graphics APIs (OpenGL/Vulkan) with the following architecture:
\begin{itemize}
    \item Scene management and object hierarchy
    \item Material and texture management system
    \item Shader compilation and management
    \item Resource allocation and memory management
\end{itemize}

\section{Technical Implementation}

\subsection{Shader System}
We have implemented a flexible shader system supporting:
\begin{itemize}
    \item Vertex shaders for vertex transformation
    \item Fragment shaders for lighting and material calculations
    \item Geometry shaders for primitive manipulation
    \item Compute shaders for parallel processing tasks
\end{itemize}

\subsection{Lighting Model}
The current implementation supports:
\begin{itemize}
    \item Phong lighting model with ambient, diffuse, and specular components
    \item Multiple light sources (point lights, directional lights)
    \item Basic shadow mapping
\end{itemize}

\subsection{Texture Mapping}
Texture support includes:
\begin{itemize}
    \item 2D texture mapping with mipmapping
    \item Texture filtering (nearest, linear)
    \item Texture coordinate generation
\end{itemize}

\section{Current Progress}

\subsection{Completed Features}
\begin{itemize}
    \item Basic rendering pipeline implementation
    \item Support for loading and rendering 3D models (OBJ format)
    \item Simple camera system with perspective projection
    \item Basic lighting and material system
    \item Texture loading and application
\end{itemize}

\subsection{Performance Metrics}
\begin{itemize}
    \item Frame rate: 60+ FPS for simple scenes
    \item Triangle throughput: ~1 million triangles per frame
    \item Memory usage: Optimized texture and buffer management
\end{itemize}

\section{Challenges and Solutions}

\subsection{Technical Challenges}
\begin{enumerate}
    \item \textbf{Memory Management}: Efficient GPU memory allocation for textures and buffers
    \item \textbf{Synchronization}: Proper synchronization between CPU and GPU operations
    \item \textbf{Shader Compilation}: Dynamic shader compilation and hot-reloading
    \item \textbf{Performance Optimization}: Minimizing draw calls and state changes
\end{enumerate}

\subsection{Implemented Solutions}
\begin{itemize}
    \item Implemented texture atlas system to reduce texture switches
    \item Used persistent mapped buffers for efficient data transfer
    \item Developed shader caching system to reduce compilation overhead
    \item Implemented instanced rendering for similar objects
\end{itemize}

\section{Next Steps}

\subsection{Short-term Goals (Next 2 weeks)}
\begin{itemize}
    \item Implement advanced lighting techniques (PBR materials)
    \item Add support for environment mapping and reflections
    \item Improve shadow quality with percentage-closer filtering
    \item Optimize rendering for complex scenes
\end{itemize}

\subsection{Long-term Goals}
\begin{itemize}
    \item Implement global illumination techniques
    \item Add support for post-processing effects
    \item Develop scene editor interface
    \item Support for animation and skeletal systems
\end{itemize}

\section{Conclusion}
We have successfully implemented a basic GPU rendering system with support for essential graphics features. The system demonstrates good performance for simple scenes and provides a solid foundation for future enhancements. Our next focus will be on improving visual quality through advanced lighting techniques and optimizing performance for more complex rendering scenarios.

\begin{acks}
We would like to thank our instructor and teaching assistants for their guidance and support throughout this project.
\end{acks}

\bibliographystyle{ACM-Reference-Format}
\bibliography{references}

\end{document}